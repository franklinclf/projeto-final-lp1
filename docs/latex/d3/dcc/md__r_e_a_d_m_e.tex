\chapter{Projeto Final de Linguagem de Programação I}
\hypertarget{md__r_e_a_d_m_e}{}\label{md__r_e_a_d_m_e}\index{Projeto Final de Linguagem de Programação I@{Projeto Final de Linguagem de Programação I}}
\label{md__r_e_a_d_m_e_autotoc_md0}%
\Hypertarget{md__r_e_a_d_m_e_autotoc_md0}%
Franklin Oliveira (20220043521)

Projeto escolhido\+: Gerenciador de Tarefas de Metodologia Ágil com Método Kanban

O gerenciador Kanban discutido anteriormente é uma aplicação de console em C++ que implementa um sistema Kanban básico. O Kanban é uma metodologia de gerenciamento visual que ajuda a controlar e otimizar fluxos de trabalho. Nesse gerenciador, você pode criar quadros, adicionar colunas e tarefas, mover tarefas entre colunas e visualizar o quadro Kanban atualizado.

A aplicação utiliza conceitos de programação orientada a objetos, como classes e estruturas de dados, para modelar o quadro Kanban, as colunas e as tarefas. Além disso, foram implementadas funcionalidades como adicionar tarefas, mover tarefas entre colunas, visualizar o quadro atual e sair do programa. 